\documentclass[12pt,letterpaper]{article}
\usepackage{fullpage}
\usepackage[top=1cm, bottom=2.5cm, left=2.5cm, right=2.5cm]{geometry}
\usepackage{amsmath,amsthm,amsfonts,amssymb,amscd}
\usepackage{lastpage}
\usepackage{enumerate}
\usepackage{fancyhdr}
\usepackage{mathrsfs}
\usepackage{xcolor}
\usepackage{graphicx}
\usepackage{listings}
\usepackage{hyperref}

\hypersetup{%
  colorlinks=true,
  linkcolor=blue,
  linkbordercolor={0 0 1}
}
 
\renewcommand\lstlistingname{Algorithm}
\renewcommand\lstlistlistingname{Algorithms}
\def\lstlistingautorefname{Alg.}

\lstdefinestyle{Python}{
    language        = Python,
    frame           = lines, 
    basicstyle      = \footnotesize,
    keywordstyle    = \color{blue},
    stringstyle     = \color{green},
    commentstyle    = \color{red}\ttfamily
}

\setlength{\parindent}{0.0in}
\setlength{\parskip}{0.05in}

% Edit these as appropriate
\newcommand\hwnumber{1}                 

\pagestyle{fancyplain}
\headheight 35pt
\lhead{\NetIDa}
\lhead{\NetIDa\\\NetIDb}
\lhead{YANZHEN LI}
\lhead{ID:40054428}
\chead{\textbf{\Large SOEN6011 Team F}}
\rhead{YANZHEN LI}
\lfoot{}
\cfoot{}
\rfoot{\small\thepage}

\headsep 1.0em
\usepackage{cite}
\usepackage{tikz}
\usepackage{color}
\begin{document}

\section*{Problem 1}

Logarithmic function : f(x) = \log_{b}X}


\begin{figure}[h]

\begin{tikzpicture}
\draw[->](-2.2,0)--(2.2,0)node[left,below,font=\tiny]{$x$};
\draw[->](0,-3.2)--(0,3.2)node[right,font=\tiny]{$y$};
\foreach \x in {-1,0,1}{\draw(\x,0)--(\x,0.05)node[below,outer sep=2pt,font=\tiny]at(\x,0){\x};}
\draw[color=blue, thick,domain=0:3]plot(\x,{log10(\x)});
\draw[color=red, thick,domain=0:3]plot(\x,{log10(\x)/log10(0.1)});
\end{tikzpicture}
\end{figure}
\hspace{0.2cm}{\qquad\color{blue} BLUE LINE}
: f(x) = \log_{10}X
{\qquad \color{red} RED LINE}
: f(x) = \log_{0.5}X\\

\hspace{0.9cm} DOMAIN:\quad b:(0,1)\cup(1,+\infty)\quad X:(0,+\infty)\\

\hspace{0.9cm} CO-DOMAIN:\quad\mathbb{R}\\
\begin{enumerate}
CHARACTERISTICS:\\
    \\
    Fixed point: Function image is always over fixed point (1,0).\\
    \\
    Monotonicity: when a\textgreater1, it is a monotonic increasing function in the 
    \\
    domain of definition.\\ 
    \\
    Parity: Non-odd and Non-even Functions\\
    \\  
    Periodicity: not a periodic function\\
    \\
    Symmetry: None\\
    \\
    Null point: X=1\\
\end{enumerate}
\end{document}
