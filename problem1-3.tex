\documentclass[12pt,letterpaper]{article}
\usepackage{fullpage}
\usepackage[top=1cm, bottom=2.5cm, left=2.5cm, right=2.5cm]{geometry}
\usepackage{amsmath,amsthm,amsfonts,amssymb,amscd}
\usepackage{lastpage}
\usepackage{enumerate}
\usepackage{fancyhdr}
\usepackage{mathrsfs}
\usepackage{xcolor}
\usepackage{graphicx}
\usepackage{listings}
\usepackage{hyperref}

\hypersetup{%
  colorlinks=true,
  linkcolor=blue,
  linkbordercolor={0 0 1}
}
 
\renewcommand\lstlistingname{Algorithm}
\renewcommand\lstlistlistingname{Algorithms}
\def\lstlistingautorefname{Alg.}

\lstdefinestyle{Python}{
    language        = Python,
    frame           = lines, 
    basicstyle      = \footnotesize,
    keywordstyle    = \color{blue},
    stringstyle     = \color{green},
    commentstyle    = \color{red}\ttfamily
}

\setlength{\parindent}{0.0in}
\setlength{\parskip}{0.05in}

% Edit these as appropriate
\newcommand\hwnumber{1}                 

\pagestyle{fancyplain}
\headheight 35pt
\lhead{\NetIDa}
\lhead{\NetIDa\\\NetIDb}
\lhead{YANZHEN LI}
\lhead{ID:40054428}
\chead{\textbf{\Large SOEN6011 Team F}}
\rhead{YANZHEN LI}
\lfoot{}
\cfoot{}
\rfoot{\small\thepage}

\headsep 1.0em
\usepackage{cite}
\usepackage{tikz}
\usepackage{color}
\begin{document}

\section*{Problem 1}

Logarithmic function : f(x) = \log_{b}X}


\begin{figure}[h]

\begin{tikzpicture}
\draw[->](-2.2,0)--(2.2,0)node[left,below,font=\tiny]{$x$};
\draw[->](0,-3.2)--(0,3.2)node[right,font=\tiny]{$y$};
\foreach \x in {-1,0,1}{\draw(\x,0)--(\x,0.05)node[below,outer sep=2pt,font=\tiny]at(\x,0){\x};}
\draw[color=blue, thick,domain=0:3]plot(\x,{log10(\x)});
\draw[color=red, thick,domain=0:3]plot(\x,{log10(\x)/log10(0.1)});
\end{tikzpicture}
\end{figure}
\hspace{0.2cm}{\qquad\color{blue} BLUE LINE}
: f(x) = \log_{10}X
{\qquad \color{red} RED LINE}
: f(x) = \log_{0.5}X\\

\hspace{0.9cm} DOMAIN:\quad b:(0,1)\cup(1,+\infty)\quad X:(0,+\infty)\\

\hspace{0.9cm} CO-DOMAIN:\quad\mathbb{R}\\
\begin{enumerate}
CHARACTERISTICS:\\
    \\
    Fixed point: Function image is always over fixed point (1,0).\\
    \\
    Monotonicity: when a\textgreater1, it is a monotonic increasing function in the 
    \\
    domain of definition.\\ 
    \\
    Parity: Non-odd and Non-even Functions\\
    \\  
    Periodicity: not a periodic function\\
    \\
    Symmetry: None\\
    \\
    Null point: X=1\\
\end{enumerate}
\section*{Problem 2}
\begin{flushleft}
		\large\textbf{1. Problem description}
	\end{flushleft}
		 Develop a Java system to calculate the result for the Logarithmic function : f(x) = \log_{b}X}.
		
	\begin{flushleft}
		\large\textbf{2. Requirements}
	\end{flushleft}
	
	\begin{itemize}
	    {a .}When the system starts, the console should display the function name and allow the user to select the logarithmic function.\\
	    \\
	    -Type attribute: Functional\\
	    \\
		{b .} The primary requirement to the function is to have only two number value as input to the function.\\
		\\
		-Type attribute: Design Constraints\\
		\\
		{c .} In case any other form of input is given, the program should prompt an effective error message to the user.\\
		\\
		-Type attribute: Functional\\
		\\
		{d .} The function accepts only a real number as its input argument. Hence, it is the responsibility of the program/function to change the illegal input to the desired input needed for it to work efficiently.\\
		\\
		-Type attribute: Design Constraints
		\\
		\\
		{e .}If the base is valid, the system should ask the user
        to input the value for variable and set it.\\
        \\
        -Type attribute: Functional\\
        \\
        {f.}If the variable is valid, the system should calculate
        the logarithm of in base without relying on java built-in functions, and store the result.\\
        \\
        -Type attribute: Functional\\
        \\
        {g .}After the calculation completes, the system should display the result on the console.\\
        \\
        -Type attribute: Functional\\
        \\
        {h. }The calculation result shall be accurate to 6 decimal places.\\
        \\
        -Type attribute: Performance\\
        
	\end{itemize}
		
	\begin{flushleft}
		\large\textbf{3. Constraints}
	\end{flushleft}
    There are few constraints that need to be followed:
		
		\begin{itemize}
			{a .} Apart from the functions related to input, output and arithmetic, use of any built-in functions provided in Java is prohibited. \\
			\\
			{b .} The domain of f(x) is b : (0,1)\cup(1,+\infty)\quad X:(0,+\infty)\\   
			\\
		\end{itemize}
	\begin{flushleft}
		\large\textbf{4. Assumptions}
	\end{flushleft}
	    \begin{itemize}
			{a .} We assume that the user interface will be text-based, depending on console input and output. \\
			\\
			{b .} User gives input for both X and a value. \\
			\\
			{c .} The 'Java system' refers to the scientific calculator\\
			\\
			{d .} Users may enter illegal characters such as letters or non-real numbers..\\
		\end{itemize}
	
	\begin{flushleft}
		\large\textbf{5. References}
	\end{flushleft}
		\begin{enumerate}
			{a .}Shapiro, J. F., & Shapiro, J. F. (1979). Mathematical programming: structures and algorithms (No. 04; QA402. 5, S4.). New York: Wiley.\\
			\\
			{b .} Riddhi, D. (2008). Beta function and its applications. The University of Tennesse, Knoxville, USA.[online] Available from: http://sces.phys.utk.edu/moreo/mm08/Riddi\\
			\\
			{c .}Olver, F. W., Lozier, D. W., Boisvert, R. F., Clark, C. W. (Eds.). (2010). NIST handbook of mathematical functions hardback and CD-ROM. Cambridge university press.
Chicago	
Chicago	
		\end{enumerate}
\end{document}
